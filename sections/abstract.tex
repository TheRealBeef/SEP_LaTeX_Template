\textbf{Abstract.}
Tekst streszczenia pracy dyplomowej należy napisać czcionką Times New Roman o rozmiarze 10 pkt z wyrównaniem dwustronnym. Między wierszami streszczenia należy stosować odstęp pojedynczy. Całość przygotowanego streszczenia winna zająć kilka stron formatu B5 (176x250 mm) (minimum 4). Marginesy: lewy prawy górny 2 cm dolny 23 cm. Zastosować formatowanie nagłówka stopki tytułu pracy oraz imiona i nazwiska autorów zgodnie z niniejszym wzorcem.

W abstrakcie umieścić zwięzły opis treści streszczenia pracy dyplomowej wraz z wnioskami końcowymi. Kilka zdań opisujących podjęte w pracy dyplomowej działania. Pokrótce opisana problematyka użyte narzędzia lub zastosowane podejście a także uzyskane efekty. Abstrakt powinien zostać napisany językiem "pop-science" aby był zrozumiały dla każdego odbiorcy "laika" zachęcając do lektury całego artykułu. W streszczeniu proszę nie cytować literatury. Objętość abstraktu nie powinna przekraczać 15 wierszy (100-150 wyrazów).